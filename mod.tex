\documentclass[2pt,-letter paper]{article}
\usepackage[left=1in, right=0.75in, top=1in, bottom=0.75in]{geometry}
\usepackage{graphicx} % Required for inserting images
\usepackage{siunitx}
\usepackage{setspace}
\usepackage{gensymb}
\usepackage{xcolor}
\usepackage{caption}
%\usepackage{subcaption}
\doublespacing
\singlespacing
\usepackage[none]{hyphenat}
\usepackage{amssymb}
\usepackage{relsize}
\usepackage[cmex10]{amsmath}
\usepackage{mathtools}
\usepackage{amsmath}
\usepackage{commath}
\usepackage{amsthm}
\interdisplaylinepenalty=2500
%\savesymbol{iint}
\usepackage{txfonts}
%\restoresymbol{TXF}{iint}
\usepackage{wasysym}
\usepackage{amsthm}
\usepackage{mathrsfs}
\usepackage{txfonts}
\let\vec\mathbf{}
\usepackage{stfloats}
\usepackage{float}
\usepackage{cite}
\usepackage{cases}
\usepackage{subfig}
%\usepackage{xtab}
\usepackage{longtable}
\usepackage{multirow}
%\usepackage{algorithm}
\usepackage{amssymb}
%\usepackage{algpseudocode}
\usepackage{enumitem}
\usepackage{mathtools}
%\usepackage{eenrc}
%\usepackage[framemethod=tikz]{mdframed}
\usepackage{listings}
%\usepackage{listings}
\usepackage[latin1]{inputenc}
%%\usepackage{color}{   
%%\usepackage{lscape}
\usepackage{textcomp}
\usepackage{titling}
\usepackage{hyperref}
%\usepackage{fulbigskip}   
\usepackage{tikz}
\usepackage{graphicx}
\lstset{
  frame=single,
  breaklines=true
}
\let\vec\mathbf{}
\usepackage{enumitem}
\usepackage{graphicx}
\usepackage{siunitx}
\let\vec\mathbf{}
\usepackage{enumitem}
\usepackage{graphicx}
\usepackage{enumitem}
\usepackage{tfrupee}
\usepackage{amsmath}
\usepackage{amssymb}
\usepackage{mwe} % for blindtext and example-image-a in example
\usepackage{wrapfig}
\graphicspath{{figs/}}
\providecommand{\cbrak}[1]{\ensuremath{\left\{#1\right\}}}
\providecommand{\brak}[1]{\ensuremath{\left(#1\right)}}
\newcommand{\sgn}{\mathop{\mathrm{sgn}}}
\providecommand{\abs}[1]{\left\vert#1\right\vert}
\providecommand{\res}[1]{\Res\displaylimits_{#1}} 
\providecommand{\norm}[1]{\left\lVert#1\right\rVert}
%\providecommand{\norm}[1]{\lVert#1\rVert}
\providecommand{\mtx}[1]{\mathbf{#1}}
\providecommand{\mean}[1]{E\left[ #1 \right]}
\providecommand{\fourier}{\overset{\mathcal{F}}{ \rightleftharpoons}}
%\providecommand{\hilbert}{\overset{\mathcal{H}}{ \rightleftharpoons}}
\providecommand{\system}{\overset{\mathcal{H}}{ \longleftrightarrow}}
 %\newcommand{\solution}[2]{\textbf{Solution:}{#1}}
%\newcommand{\solution}{\noindent \textbf{Solution: }}
\newcommand{\cosec}{\,\text{cosec}\,}
\providecommand{\dec}[2]{\ensuremath{\overset{#1}{\underset{#2}{\gtrless}}}}
\newcommand{\myvec}[1]{\ensuremath{\begin{pmatrix}#1\end{pmatrix}}}
\newcommand{\myaugvec}[2]{\ensuremath{\begin{amatrix}{#1}#2\end{amatrix}}}
\newcommand{\mydet}[1]{\ensuremath{\begin{vmatrix}#1\end{vmatrix}}}
\title{MATHEMATICS}
\author{SECTION A}
\date{\today}
\begin{document}

\maketitle

\begin{enumerate}
\section{Matrix}
\item Find $|AB|$,if $A = \myvec{0 & -1 \\ 0 & 2}$ and $B = \myvec{3 & 5 \\ 0 & 0}$.
\item If $A = \myvec{p & 2 \\ 2 & p}$ and $|A^3| = 125$
, then find values of p.
\item Using properties of determinants, show that      $\begin{vmatrix}                                   
3a & -a + b & -a + c \\ -b + a & 3b & -b + c \\-c + a & -c + b & 3c\end{vmatrix}$ = 3 \brak{a + b + c}\brak{ab + bc + ca}
\item Find the inverse of the following matrix, using elementary transformations : $\myvec{2 & 0 & -1 \\ 5 & 1& 0 \\ 0 & 1 & 3}$
\section{Differentiation}
\item Differentiate $e^{\sqrt{3x}}$, with respect to ${x}$.
\item Find the order and degree if defined of the differential equation. 
\begin{align*}
 \dfrac{d^2y}{d^2x}+x\brak{\dfrac{dy}{dx}}^2=2x^2\log\brak{\dfrac{d^2y}{dx^2}}
\end{align*}
\item Find the general solution of the differential equation $\frac{dy}{dx} = e^{x+y}$.
\item Solve the differential equation :                \begin{align*}  x\frac{dy}{dx} = y - x\tan{\brak{\frac{y}{x}}}  \end{align*}
\item If $x\sqrt{1+y}$+ $y\sqrt{1+x}$=0 and $x\neq y$,   prove that $\dfrac{dy}{dx} = -\frac{1}{\brak{x+1}^2}$
\item If $\brak{\cos x}^y = \brak{\sin y}^x$, find $\frac{dy}{dx}$.
\section{Vectors}
\item A line passes through the point with position vector $2\hat{i} - \hat{j} + 4\hat{k}$ and is in the direction of the vector $\hat{i} + \hat{j} -2\hat{k}.$ Find the equation of the line in the cartesian form.
\item $\mydet{\overrightarrow{a}}= 2$, $\mydet{\overrightarrow{b}} = 7$ and  $\overrightarrow{a} \times \overrightarrow{b} = 3\hat{i} + 2\hat{j} + 6\hat{k}$, find the angle between $\mydet{a}$ and $\mydet{b}$.
\item Find the volume of a cuboid whose edges are given by $-3\hat{i} + 7\hat{j} + 5\hat{k}$, $-5\hat{i} + 7\hat{i} - 3\hat{k}$ and $7\hat{i} -5\hat{j} -3\hat{k}$.
\item Find the direction cosines of a line which makes equal angles with the coordinate axes.
\item Find the cartesian and vector equations of the plane passing through the points $A\brak{2,5,-3}$, $B\brak{-2,-3,5}$, $C\brak{5,3,-3}$.
\item Find the equation of the line passing through $\brak{2, 0, 3}$, $\brak{1, 1, 5}$ and $\brak{3, 2, 4}$.Also, find their point of intersection.
\section{Optimization}
\item The volume of a cube is increasing at the rate of $8 cm^3/s$. How fast is the surface area increasing when the length of its edge is $12$ cm ?
\section{Functions}
\item Let $f : N \rightarrow Y$ be a function defined as 
\begin{equation}
 f(x) = 4x + 3,
\end{equation}
Where $Y = {y \in N : y = 4x + 3, for some x \in N}$. Show that f is invertible. Find its inverse.
\section{Integration}
\item Find: \begin{align*}\int{\frac{x^2+x+1}{\brak{x+2}\brak{x^2+1}}}dx\end{align*}		
\item Find :                                            \begin{align*} \int{\sqrt{3 - 2x - x^2}}{dx} \end{align*}
\item Find: $\int_{1}^{3}{\brak{x^2+2+e^{2x}}}dx$ as the limit of sums.           
\item Using integration, find the area of the triangular region whose sides have the equations $y = 2x + 1$, $y= 3x + 1$, $x = 4$.
\item If $\brak{a + bx}e^\frac{y}{x}=x$, then prove that\begin{align*}                                         x^3\frac{d^2y}{dx^2} = \brak{x\frac{dy}{dx}-y}^2. \end{align*}
\section{Probability}
\item A coin is tossed 5 times. what is the probablity     of getting ({i}) 3 heads,({ii}) at most 3 heads.
\item There are three coins. One is a two-headed coin,  another is a baised coin that comes up heads $75\%$ of the time and the third is an unbaised coin.One of the three coins is chosen at random and tossed. If it shows heads,What is the probability that it is the two-headed coin ?
\item Find the probability distribution of $X$, the number of heads in a simultaneous toss of two coins.
\section{Intersection of conics}
\item Find the point on the curve $y^2 = 4x$, which is nearest to the point $\brak{2,-8}$.
\end{enumerate}
\end{document}

 
